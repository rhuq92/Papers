\newcommand{\pt}[1]{\textcolor{red}{#1}}
\documentclass[12pt,reqno]{amsart}
\usepackage{amsmath}
\usepackage{cancel}
\usepackage{amssymb}
\usepackage[left=3cm,top=3cm,right=3cm,bottom=3cm]{geometry}
\usepackage{epsfig}
\usepackage[normalem]{ulem}
\usepackage[usenames,dvipsnames]{color}
\usepackage[textwidth=20mm]{todonotes}
\usepackage{hyperref}
\usepackage{eqnarray}
\usepackage{mathtools}
\usepackage{mathrsfs}
\newtheorem{case}{Case}
\usepackage{algpseudocode}
\usepackage{algorithm}
\usepackage{tcolorbox}
\usepackage{hyperref}


\usepackage{soul}
\usepackage{tikz}
\usetikzlibrary{backgrounds}
\usetikzlibrary{intersections}

%\usepackage{pgfplots}
\usepackage{float}
\usepackage{xcolor}
\colorlet{shadecolor}{pink!69}
%\usepackage{floatrow}
% Table float box with bottom caption, box width adjusted to content. 
%Using floatrow instead broke my algorithm

\newtheorem{theorem}{Theorem}[section]
\newtheorem{lemma}[theorem]{Lemma}

\newcommand{\Gnp}{\G(n,p)}

\newcommand{\fancyP}{\mathscr{P}}

\newcommand{\Cond}[2]{\bigm( {#1} \bigm\vert {#2}\bigm)}
\newcommand{\G}{\mathbb{G}}
\newcommand{\FancyG}{\mathcal{G}}
\newcommand{\lineG}{\mathbb{G}}
\newcommand{\Ebf}{\textit{{\textbf{E}}}}
\newcommand\eps{\varepsilon}
\newcommand{\E}{\mathbb E}
\newcommand{\Z}{\mathbb Z}
\newcommand{\Cov}{\mathbb C\textrm{ov}}
\newcommand{\Var}{\mathbb V\textrm{ar}}
\newcommand{\Prob}{\mathbf{P}}
\newcommand{\Bin}{\mathrm{Bin}}
\newcommand{\Geom}{\mathrm{Geom}}
\newcommand{\Nn}{{\mathbb N}}
\DeclarePairedDelimiter\ceil{\lceil}{\rceil}
\DeclarePairedDelimiter\floor{\lfloor}{\rfloor}
\newcommand{\rh}[1]{\textcolor{blue}{#1}}
\usepackage{framed}

%usage: \begin{frshaded} \end{frshaded}


\title{Policy Gradient Methods, Curvature, and Distribution Shift}

\date{November 2, 2020}
\author{Sham Kakade}

\begin{document}
\maketitle

These are notes from a talk given by Dr. Sham Kakade on November 2nd on some rather high-level concepts in reinforcement learning. The talk can be accessed 
\href{http://www.fields.utoronto.ca/talks/Policy-Gradient-Methods-Curvature-and-Distribution-Shift}{\textbf{here}}.

In this write-up, I'll provide a very brief conceptual overview of some ideas that Dr. Kakade deals with before diving into his talk.

\section{What is reinforcement learning?}
\textbf{Reinforcement learning} is an area of machine learning that is concerned with the behaviour of agents who are concerned with maximizing utility. In a sense, it is paradigmatically divergent from \textbf{supervised} \& \textbf{unsupervised} learning, which focus upon recognizing patterns within a given domain.

Reinforcement learning is studied by scholars schooled in a multitude of different fields. Game theorists, control theorists, researchers in operations research, and statisticians, among others, study reinforcement learning. 

Basic reinforcement is modeled as a Markov decision process (MDP), consisting of four elements:
\begin{itemize}
\item A set of environment and agent states, $\mathcal{S}$. Can be either finite or infinite. We assume it's finite or countably infinite.
\item A set of actions $\mathcal{A}$ of an agent. Can be either finite or finite, but we typically assume it's finite.
\item A transition function $P :\mathcal{S} \times \mathcal{A} \rightarrow \delta(\mathcal{S})$, where $\delta(\mathcal{S})$ is the space of probability distributions over $\mathcal{S}$. $P \Cond{s'}{s,a}$. In other words, it's the \textit{probability simplex}.
\item $P_{a}(s,s') = \Pr\left(\mathcal{S}_{t+1}=s' \bigm\vert \mathcal{S}_{t} = s, a_{t} = a \right)$
\item A feward function $r: \mathcal{S} \times \mathcal{A} \rightarrow [0,1]$. This is the agent's reward for taking action $a$ at state $s$.
\item A discount factor $\gamma \in [0,1)$, which defines a horizon for the problem. Intuitively, $\gamma$ considers the tiem-scale of the pay-off (are we rewarding delayed gratification? In which case, $\gamma$ will be large. Are we rewarding immediate gratification? In which case, $\gamma$ will be small.)
\item An initial state distribution $\mu \in \delta(\mathcal{S})$, which specifies how the initial state $s_{0}$ is generated.
\end{itemize}

A Markov decision process occurs over discrete steps of time $t$ with can be characterized with a state $s_{t}$ and an agent's action $a_{t}$. The goal of the agent is to learn a policy $\pi: A \times S \rightarrow [0,1]$, $ \pi(a,s) = \Pr\Cond{a_{t} = a}{s_{t} = s} $
\section{What are policy methods?}
In a given MDP $M = (\mathcal{S},\mathcal{A}, P, r, \gamma, \mu)$, the agent interacts with the environment according to the following protocol: the agent starts at $s_{0} ~ \mu$ \& at each time step $t = 0,1,2,\ldots,$ the agent takes an action $a_{t} \in \mathcal{A}$, obtains the immediate reward $r_{t}(s_{t},a_{t})$, and observes the next state $s_{t+1}$ sampled according to $s_{t+1} ~ P \Cond{\cdot}{s_{t},a_{t}}$

The interaction record $\tau$ at time $t$,
\begin{equation*}
\tau_{t} = (s_{0},a_{0},r_{1} \ldots s_{t})5
\end{equation*}
is called the trajectory at time $t$ and includes the state at time $t$, $s_{t}$.

A policy specifies a decision-making strategy in which the agent is informed by the history of their observations (in this sense, a Markov decision process should be considered conceptually different from a Markov chain in terms of agnosticity to history). 

A policy is a (possibly randomized) mapping from a trajectory to an action. Thus,
\begin{equation*}
\pi: \mathcal{H} \rightarrow \delta(\mathcal{A})
\end{equation*}
, where $\mathcal{H}$ is the set of all possible trajectories (of all lengths) and $\delta \mathcal{A}$ is the space of all probability distributions over $\mathcal{A}$. A stationary policy $\pi : \mathcal{S} \rightarrow \mathcal{A}$ specifies a decision making strategy based solely on the current state (that is $a_{t} \sim \pi \Cond{\cdot}{s_{t}} $). A determinstic stationary policy is of the form $\pi: \mathcal{S} \rightarrow \mathcal{A}$

Now, we define \textbf{values} for general policies. For a fixed policy and a starting state $s_{0 = s}$, we define the value function $V_{m}^{\pi}: \mathcal{S} \rightarrow \mathbb{R}$ as the discounted sum of future rewards
\begin{equation}\label{eqn:Values}
V_{m}^{\pi}(s) = \E\left[ \sum\limits_{t = 0}^{\infty} \Cond{\gamma^{t}r(s_{t},a_{t})}{\pi, s_{0} = s}   \right]
\end{equation}

Where the expectation is computed with respect to the randomness of the trajectory, that is, the randomness in state-transitions and stochasticity of $\pi$. Since $r(s,a)$ is bounded between 0 \& 1, we have $0 \leq V_{m}^{\pi}(s) \leq 1/(1 - \gamma)$

Similarly, the action-value (or Q-value) function $Q_{M}^{\pi} : \mathcal{S} \times \mathcal{A} \rightarrow \mathbb{R}$ is defined as 
\begin{equation}\label{eqn:Q}
Q_{m}^{\pi} = \E \left[\sum\limits_{t = 0}^{\infty} \Cond{r(s_{t},a_{t})}{\pi, s_{0} = s, a_{0} = a}    \right]
\end{equation}
, which is also bounded from below by 0 and above by $1/(1-\gamma)$

\medskip

The optimization that the agent aims to solve is to maximize $V_{M}^{\pi}$ by finding a policy $\pi$. The agent thus wants to maximize
\begin{equation}\label{eqn:optvalues}
\max\limits_{\pi} V_{M}^{\pi}(s)
\end{equation}

\section{What was Dr. Kakade talking about?}



\begin{thebibliography}{99}

%\bibitem{label} #name, #titleofwork, #publisher, #edition, #source
\bibitem{label} A.\ Agarwal, N.\ Jiang, S.M.\ Kakade, W.\ Sun; Reinforcement Learning: Theory and Algorithms; accessed October 27, 2020; available at \texttt{https://rltheorybook.github.io/}
\end{thebibliography}
\end{document}